\documentclass[12pt]{article}

\usepackage{graphicx}
\usepackage[hypertexnames=false,colorlinks=true,breaklinks]{hyperref}
\usepackage{tikz}
\usepackage{verbatim}
\usepackage{natbib}
\usepackage{apalike}
\usepackage{amsmath,amssymb}
\usepackage{endfloat}

\newcommand{\estNumber}{82227365}

\title{First LDS analysis of Emmett' data}
\author{Joaqu\'{i}n Rapela\thanks{j.rapela@ucl.ac.uk}}

\begin{document}

\maketitle

\section{Introduction}

Here we use a Linear Dynamical System
model (LDS)\footnote{\url{https://github.com/joacorapela/ssm}} to explore the low dimensional
structure of Emmett's switching-task recordings~\citep{thompsonEtAl24}.

\section{Methods}

\subsection{Data}

We analyzed two minutes of continuous electrophysiological recordings starting
at (electrophysiological) time 5512 from
\texttt{EJT178\_implant1/recording1\_15\-03\-2022}, and neurons indices 124 to
223 in the striatum.
%
Figure~\ref{fig:spikeTimes} plots the spike times used in this analysis.

\begin{figure}
    \centering
    \href{http://www.gatsby.ucl.ac.uk/~rapela/emmett/reports/firstLDSreport/figures/spike_times_start5512.00_end5632.00.html}{\includegraphics[width=6in]{../../../figures/spike_times_start5512.00_end5632.00.png}}

    \caption{Spikes times used in this analysis: we analyzed two minutes of
    continuous electrophysiological recordings starting at
    (electrophysiological) time 5,512 from
    \texttt{EJT178\_implant1/recording1\_15\-03\-2022}, and neurons indices
    124 to 223 in the striatum.
    %
    Click on the image to get its interactive version.
    %
    }

    \label{fig:spikeTimes}
\end{figure}

\subsection{LDS model}

The LDS model estimates $K$ latent variables, $x_0[n],\ldots,x_{K-1}[n]$,
$n\in\{0,\ldots,N-1\}$.  We
concatenated the latent variables into a latent vector $\mathbf{x}[n]$:

\begin{align}
    \mathbf{x}[n]=\left[\begin{array}{c}
        x_0[n]\\
                            \vdots\\
                            x_{K-1}[n]
                        \end{array}\right]
    \label{eq:latentsVector}
\end{align}

The latents evolve in time following a first-order Markov process with additive
Gaussian noise:

\begin{align*}
    \mathbf{x}[n]=A\mathbf{x}[n-1]+\mathbf{v}[n]\quad\text{with}\ \mathbf{v}[n]\sim\mathcal{N}(0, Q)
\end{align*}

\noindent starting from
$\mathbf{x}_0\sim\mathcal{N}(\mathbf{x}_0|\mathbf{m}_0,V_0)$.

The latent variables are combined linearly to approximate the firing rate of
the recorded neurons (bin size 20~ms), $\mathbf{y}[n]$, with a constant offset,
$\mathbf{a}$, and an additive noise, $\mathbf{w}[n]$.

\begin{align*}
    \mathbf{y}[n]=\mathbf{a}+C\mathbf{x}[n]+\mathbf{w}[n]\quad\text{with}\ \mathbf{w}[n]\sim\mathcal{N}(0, R)
\end{align*}

We estimated an LDS models with $K=5$ latent variables
(Eq.~\ref{eq:latentsVector}; the selection of the number of latent variables
was arbitrary, and we should perform model selection in later iterations).

We estimate  the model parameter $\{\mathbf{m}_0, V_0, \mathbf{u}, A, Q,
\mathbf{a}, Z, R\}$ using the expectation maximisation algorithm
(\href{https://github.com/joacorapela/lds_neuralLatents_striatum/blob/master/code/scripts/doEstimateEM.py}{doEstimateEM.py}).
%
We then used the Kalman Filter algorithm, with these parameters, to infer the
latents variables, $\mathbf{x}[n]$ in Equation~\ref{eq:latentsVector}
(\href{https://github.com/joacorapela/lds_neuralLatents_striatum/blob/master/code/scripts/doKalmanFilteringWithEstimatedParams.py}{doKalmanFilteringWithEstimatedParams.py}).
%
We also forecasted firing rates of individual neurons
(\href{https://github.com/joacorapela/lds_neuralLatents_striatum/blob/master/code/scripts/doPlotObservationsForecasts.py}{doPlotObservationsForecasts.py}).

\section{Results}

\subsection{Simultaneous spikes across the striatum}

The inferred latent variables are shown in Figure~\ref{fig:inferredLatents}.
%
Latent~0, that captures most of variance in the spike rates, shows large
negative deflections every 1 to 3~seconds.
Figures~\ref{fig:inferredLatentsZoom1} and~\ref{fig:inferredLatentsZoom2} show
zoomed-in views of the latents, and Figures~\ref{fig:spikeTimesZoom1}
and~\ref{fig:spikeTimesZoom2} show corresponding zoomed-in views of the spikes
times. We note that \textbf{at times of large negative deflections of latent~0
there is a large number of simultaneous spikes across many neurons in the
striatum}.

\begin{figure}
    \centering
    \href{http://www.gatsby.ucl.ac.uk/~rapela/emmett/reports/firstLDSreport/figures/91973683_state_filtered_from5512.00_to5632.00.html}{\includegraphics[width=6in]{../../../figures/91973683_state_filtered_from5512.00_to5632.00.png}}
    \caption{Orthonormalised latent variables ($\mathbf{x}(t)$ in
    Eq.~\ref{eq:latentsVector}) inferred by the Kalman filter algorithm.
    %
    The colored vertical lines indicate `poke\_in` times, with
    blue, red, cyan, yellow, purple, green and magenta corresponding to ports
    1 to 7, respectively.
    %
    The correct sequence is $2\rightarrow 1\rightarrow 6\rightarrow 3\rightarrow 7$
    corresponding to
    $\text{red}\rightarrow\text{blue}\rightarrow\text{green}\rightarrow\text{cyan}\rightarrow\text{magenta}$.
    %
    Note the sharp downward peaks of latent 0 (the latent variable capturing
    most of the variance in the response). Zoomed versions of this image appear
    in Figures~\ref{fig:inferredLatentsZoom1}
    and~\ref{fig:inferredLatentsZoom2}.
    %
    Click on the image to get its interactive version.
    %
    }
    \label{fig:inferredLatents}
\end{figure}

\begin{figure}
    \centering
    \includegraphics[width=6in]{../../../figures/91973683_state_filtered_from5512.00_to5632.00_zoomed1.png}

    \caption{Zoomed-in view one of the latents in Figure~\ref{fig:inferredLatents}}.

    \label{fig:inferredLatentsZoom1}
\end{figure}

\begin{figure}
    \centering
    \includegraphics[width=6in]{../../../figures/91973683_state_filtered_from5512.00_to5632.00_zoomed2.png}

    \caption{Zoomed-in view two of the latents in Figure~\ref{fig:inferredLatents}}.

    \label{fig:inferredLatentsZoom2}
\end{figure}

\begin{figure}
    \centering
    \includegraphics[width=6in]{../../../figures/spike_times_start5512.00_end5632.00_zoomed1_marked.png}

    \caption{Zoomed-in view of the spikes-times in Figure~\ref{fig:spikeTimes}
    with bounding boxes around the times of large negative deflections of
    latent~0 in Figure~\ref{fig:inferredLatentsZoom1}.}

    \label{fig:spikeTimesZoom1}
\end{figure}

\begin{figure}
    \centering
    \includegraphics[width=6in]{../../../figures/spike_times_start5512.00_end5632.00_zoomed2_marked.png}

    \caption{Zoomed-in view of the spikes-times in Figure~\ref{fig:spikeTimes}
    with bounding boxes around the times of large negative deflections of
    latent~0 in Figure~\ref{fig:inferredLatentsZoom2}.}

    \label{fig:spikeTimesZoom2}
\end{figure}

\subsection{Forecasting spikes rates}

Figures~\ref{fig:forecastingH0.02}-\ref{fig:forecastingH1.00} show model
forecastings of spike rates, from a sample cluster, with horizon ranging from
20~ms to 1.0~secs into the future. In these figures the forecasted time series
is plotted at forecasting times. This means that if the forecasting horizon was
$h$~seconds, the forecasted trace at time $t$ represents the forecasted spike
rate at time $t$, that was generated using data up to time $t-h$.

\begin{figure}
    \centering
    \href{http://www.gatsby.ucl.ac.uk/~rapela/emmett/reports/firstLDSreport/figures/91973683_obs_forecasts_horizon0.020_cluster179_from5512.00_to5632.00.html}{\includegraphics[width=6in]{../../../figures/91973683_obs_forecasts_horizon0.020_cluster179_from5512.00_to5632.00.png}}

    \caption{Forecasted spikes rates of example cluster 179 with an horizon
    $h=0.02$~seconds.
    %
    The forecasted time series is plotted at forecasting times. This means that
    if the forecasting horizon was $h$~seconds, the forecasted trace at time
    $t$ represents the forecasted spike rate at time $t$, that was generated
    using past data up to time $t-h$.
    %
    Click on the image to get its interactive version.
    %
    }
    \label{fig:forecastingH0.02}
\end{figure}

\begin{figure}
    \centering
    \href{http://www.gatsby.ucl.ac.uk/~rapela/emmett/reports/firstLDSreport/figures/91973683_obs_forecasts_horizon0.100_cluster179_from5512.00_to5632.00.html}{\includegraphics[width=6in]{../../../figures/91973683_obs_forecasts_horizon0.100_cluster179_from5512.00_to5632.00.png}}

    \caption{Forecasted spikes rates of example cluster 179 with an horizon
    $h=0.10$~seconds.
    %
    Same format as in Figure~\ref{fig:forecastingH0.02}.
    %
    Click on the image to get its interactive version.
    %
    }
    \label{fig:forecastingH0.10}
\end{figure}

\begin{figure}
    \centering
    \href{http://www.gatsby.ucl.ac.uk/~rapela/emmett/reports/firstLDSreport/figures/91973683_obs_forecasts_horizon0.500_cluster179_from5512.00_to5632.00.html}{\includegraphics[width=6in]{../../../figures/91973683_obs_forecasts_horizon0.500_cluster179_from5512.00_to5632.00.png}}

    \caption{Forecasted spikes rates of example cluster 179 with an horizon
    $h=0.50$~seconds.
    %
    Same format as in Figure~\ref{fig:forecastingH0.02}.
    %
    Click on the image to get its interactive version.
    %
    }
    \label{fig:forecastingH0.50}
\end{figure}

\begin{figure}
    \centering
    \href{http://www.gatsby.ucl.ac.uk/~rapela/emmett/reports/firstLDSreport/figures/91973683_obs_forecasts_horizon1.000_cluster179_from5512.00_to5632.00.html}{\includegraphics[width=6in]{../../../figures/91973683_obs_forecasts_horizon1.000_cluster179_from5512.00_to5632.00.png}}

    \caption{Forecasted spikes rates of example cluster 179 with an horizon
    $h=1.00$~seconds.
    %
    Same format as in Figure~\ref{fig:forecastingH0.02}.
    %
    Click on the image to get its interactive version.
    %
    }
    \label{fig:forecastingH1.00}
\end{figure}

The forecasting model is learning two interesting global trends in the
recordings.
%
First, when repeated firing is detected, the model learns that the firing rate
should first increase to a peak value, and then decrease to baseline.
%
Second, the model learns that increases in firing  rate are regular, and
sometimes forecasts increases in firing rate, even without previous spiking.
%
This features are more prominent for horizons $h=0.02, 0.1$~sec. Larger
horizons of $h=0.5 and 1.0$~sec show delays.
%
However, we will perform a proper evaluation of forecasting in the near future.

\section{Future work}

\begin{enumerate}

    \item Analyse other time periods from 
    \texttt{EJT178\_implant1/recording1\_15\-03\-2022}.

    \item Analyze other animals.

    \item Improve the evaluation of forecasters.

    \item Perform real-time latent visualisation with unsorted spikes.

\end{enumerate}

\bibliographystyle{apalike}
\bibliography{replay.bib}

\end{document}
